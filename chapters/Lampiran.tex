\appendix
\chapter*{LAMPIRAN}
\addcontentsline{toc}{chapter}{LAMPIRAN}
\setcounter{section}{0} % Mengatur ulang penomoran section
\renewcommand{\thesection}{\Alph{section}}
\renewcommand{\thesubsection}{\arabic{subsection}}
\setcounter{page}{1}

\textbf{\textit{Lampiran 1}}\\

\begin{center}
        \textbf{\MakeUppercase{\large{rencana anggaran biaya}}}\\
        \textbf{\MakeUppercase{\normalsize{\judulproyek}}}\\
\end{center}

\section{Pemasukan}


\section{Pengeluaran}
\begin{table}[H]
	\centering
	
	\label{Harga Komponen}
	\begin{tabularx}{\linewidth}{
			|p{\dimexpr.02\linewidth-2\tabcolsep-1.3333\arrayrulewidth}% column 1
			|p{\dimexpr.255\linewidth-2\tabcolsep-1.3333\arrayrulewidth} % column 2
			|p{\dimexpr.21\linewidth-2\tabcolsep-1.3333\arrayrulewidth}% column 3
			|p{\dimexpr.11\linewidth-2\tabcolsep-1.3333\arrayrulewidth}% column 4
			|p{\dimexpr.11\linewidth-2\tabcolsep-1.3333\arrayrulewidth}% column 5
			|p{\dimexpr.21\linewidth-2\tabcolsep-1.3333\arrayrulewidth}|% column 6
		}
		\hline
		\rowcolor{blue!20} % Pewarnaan baris
		\multicolumn{1}{|c|}{\textbf{No.}} & 
		\multicolumn{1}{|c|}{\textbf{Komponen}} &
		\multicolumn{1}{c|}{\textbf{Harga Satuan}} & 
		\multicolumn{1}{c|}{\textbf{Jumlah}} &
		\multicolumn{1}{c|}{\textbf{Satuan}} &
		\multicolumn{1}{c|}{\textbf{Total}}\\ \hline
		1 & NA226 (Voltage Sensor 0-36V) & Rp 56.000,00 & 1 & unit  & Rp 56.000,00  \\ \hline
		2 & ACS758 (Current Sensor max 50A)&  Rp 63.000,00 & 1 & unit  & Rp 63.000,00  \\\hline
		3 & ESP32  & Rp 60.000,00  & 1 & unit &  Rp 60.000,00  \\ \hline
		4 &  Resistor &  & 1 & unit &  \\ \hline
		5 &  Induktor &  & 1 & unit &  \\ \hline
		6 &  Kapasitor &  & 1 & unit &  \\ \hline
		7 &  MOSFET IRFP250 & Rp 25.000,00 & 1 & unit & Rp 25.000,00 \\ \hline
		8 &  Dioda MBR1045 & Rp 10.000,00 & 1 & unit & Rp 10.000,00 \\ \hline
		9 &  Box Hitam &  & 1 & unit & \\ \hline
		10 &  PCB & Rp 6.000,00 & 1 &10 cm x 20 cm & Rp 6.000,00\\ \hline
		
		
	\end{tabularx}
\end{table}