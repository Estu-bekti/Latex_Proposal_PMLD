\renewcommand{\thesection}{\Roman{section}}

\section{Latar Belakang}

Pada era modern ini, kebutuhan akan efisiensi energi dalam sistem elektronik semakin meningkat. Salah satu tantangan utama dalam sistem daya adalah bagaimana mengonversi tegangan dengan efisiensi tinggi dan kehilangan daya seminimal mungkin. Salah satu solusi yang umum digunakan dalam dunia elektronika adalah buck converter, sebuah konverter daya DC-DC yang berfungsi untuk menurunkan tegangan dari sumber ke level yang lebih rendah sesuai kebutuhan beban.

Dalam berbagai aplikasi, seperti sistem tenaga surya, kendaraan listrik, dan perangkat elektronik portabel, penggunaan buck converter menjadi sangat penting. Namun, masih banyak tantangan dalam mendesain buck converter yang memiliki efisiensi tinggi, ukuran yang kompak, serta biaya yang terjangkau. Oleh karena itu, penelitian dan pengembangan dalam bidang ini terus dilakukan untuk mendapatkan desain yang optimal.

Melalui proyek mandiri lintas disiplin ini, kami berupaya merancang dan mengimplementasikan buck converter yang dapat digunakan dalam berbagai aplikasi elektronika dengan mempertimbangkan efisiensi, stabilitas tegangan, serta kemudahan dalam produksi. Dengan adanya proyek ini, kami berharap dapat memberikan kontribusi dalam pengembangan teknologi konversi daya sekaligus meningkatkan pemahaman dan keterampilan mahasiswa dalam bidang elektronika daya dan instrumentasi.

\section{Rumusan Masalah}
Perancangan sistem konversi daya, efisiensi dan kestabilan tegangan menjadi faktor utama yang harus diperhatikan. Buck converter sebagai salah satu jenis konverter DC-DC memiliki peran penting dalam berbagai aplikasi elektronik, namun masih terdapat beberapa tantangan dalam proses perancangannya.

Dalam implementasinya, terdapat berbagai tantangan yang harus diatasi, termasuk upaya dalam meningkatkan efisiensi daya dan menjaga kestabilan tegangan output. Peningkatan efisiensi daya dan kestabilan tegangan output dapat dipengaruhi oleh komponen yang digunakan dalam rangkaian buck converter. Penentuan parameter komponen yang digunakan memerlukan perhitungan yang optimal untuk mencapai performa yang terbaik.  Oleh karena itu, penelitian ini berfokus pada perancangan buck converter dengan efisiensi tinggi dan kestabilan tegangan yang baik serta penentuan parameter komponen yang optimal guna meningkatkan kinerja keseluruhan sistem.


\section{Tujuan }
Berikut adalah beberapa tujuan penelitian yang telah ditetapkan untuk memandu jalannya penelitian ini: 
\begin{enumerate}
    \item Merancang dan mengembangkan buck converter yang mampu menurunkan tegangan DC dengan efisiensi tinggi dan kestabilan yang baik.
    \item Mengoptimalkan efisiensi daya pada sistem konversi tegangan untuk mengurangi kehilangan energi.
    \item Mendukung pengembangan teknologi konversi daya yang dapat diterapkan pada berbagai bidang seperti sistem tenaga surya.
\end{enumerate}

\section{Kebutuhan}
Dalam perancangan buck converter, diperlukan berbagai komponen utama yang mendukung fungsi sistem agar dapat bekerja dengan optimal. Beberapa komponen yang dibutuhkan dalam proyek ini antara lain:
\begin{enumerate}
    \item INA226 (Voltage Sensor 0-36V)
    \item ACS758 (Current Sensor max 50A)
    \item ESP32
    \item Resistor
    \item Induktor
    \item Kapasitor
    \item MOSFET IRFP250
    \item Dioda MBR1045
    \item Baterai 18650
    \item Box hitam
    \item PCB
\end{enumerate}

\section{Target Luaran}
\begin{enumerate}
	\item Program
	\item Prototype
	\item Banner
	\item Laporan akhir
\end{enumerate}

\section{Hipotesis}
\begin{enumerate}
	\item Penerapan metode kontrol PID (Proportional-Integral-Derivative) pada sistem buck converter berbasis mikrokontroler Arduino dapat meningkatkan kinerja regulasi tegangan output dengan menghasilkan respon transien yang lebih cepat, mengurangi kesalahan steady-state, serta meningkatkan kestabilan sistem terhadap perubahan beban maupun variasi tegangan input, dibandingkan dengan buck converter yang tidak menggunakan sistem kontrol aktif.
	\item Penerapkan metode kontrol PID (Proportional-Integral-Derivative) yang diimplementasikan melalui mikrokontroler Arduino pada rangkaian buck converter, maka sistem akan mampu menghasilkan tegangan output yang lebih stabil, respons dinamis yang lebih cepat terhadap perubahan beban, serta mengurangi kesalahan steady-state dibandingkan dengan sistem buck converter tanpa kontrol PID.
\end{enumerate}

\section{Timeline}
\begin{table}[H]
	
	\centering
	\renewcommand{\arraystretch}{1.5}
	\setlength{\tabcolsep}{4pt}
	\caption{Timeline Kegiatan Proyek}
	\resizebox{\textwidth}{!}{
		\begin{tabular}{|c|p{6cm}|*{14}{>{\centering\arraybackslash}p{0.5cm}|}p{3.5cm}|}
			\hline
			\textbf{No.} & \textbf{Kegiatan} & \multicolumn{14}{c|}{\textbf{Pertemuan}} & \textbf{Nama anggota terlibat} \\
			\cline{3-16}
			& & 1 & 2 & 3 & 4 & 5 & 6 & 7 & 8 & 9 & 10 & 11 & 12 & 13 & 14 & \\
			\hline
			1 & Mahasiswa mendapatkan pemberitahuan mengenai pelaksanaan mata kuliah Proyek Mandiri Lintas Disiplin (PMLD), termasuk penjelasan mengenai tujuan, mekanisme kerja, serta luaran yang diharapkan dari kegiatan ini. & \cellcolor{yellow} &  &  &  &  &  &  &  &  &  &  &  &  &  & Estu, Fuad, Desta, Enggar, Ali,Rosus \\
			\hline
			2 & Proses pembentukan tim yang terdiri dari mahasiswa lintas jurusan, guna menunjang kerja kolaboratif dalam menyelesaikan proyek yang dirancang. &  & \cellcolor{yellow} &  &  &  &  &  &  &  &  &  &  &  &  & Estu, Fuad, Desta, Enggar, Ali,Rosus \\
			\hline
			3 & Melakukan perkenalan antar anggota dan pengenalan awal mengenai konsep dasar buck converter sebagai fokus utama proyek. &  &  & \cellcolor{yellow} &  &  &  &  &  &  &  &  &  &  &  & Estu, Fuad, Desta, Enggar, Ali,Rosus \\
			\hline
			4 & Praktik pembuatan rangkaian buck converter sederhana menggunakan software MATLAB sebagai langkah awal untuk memahami prinsip kerja dan karakteristik dasar dari alat tersebut. &  &  &  & \cellcolor{yellow} &  &  &  &  &  &  &  &  &  &  & Estu, Fuad, Desta, Enggar, Ali,Rosus \\
			\hline
			5 & Melakukan eksplorasi terhadap penggunaan kontrol PID untuk meningkatkan performa buck converter yang telah dirancang. &  &  &  &  & \cellcolor{yellow} &  &  &  &  &  &  &  &  &  & Estu, Fuad, Desta, Enggar, Ali,Rosus \\
			\hline
			6 & Menyusun dan menyempurnakan proposal proyek &  &  &  &  & \cellcolor{yellow} & \cellcolor{yellow} & \cellcolor{yellow} &  &  &  &  &  &  &  & Estu, Fuad, Desta, Enggar, Ali,Rosus \\
			\hline
			7 & Ujian Tengah Semester (UTS) &  &  &  &  &  &  &  & \cellcolor{yellow} &  &  &  &  &  &  & Estu, Fuad, Desta, Enggar, Ali,Rosus \\
			\hline
		\end{tabular}
	}
\end{table}
\section{Rencana Anggaran Biaya}
Terlampir (Lampiran 1) 

\section{Susunan Kelompok}
Terlampir (Lampiran 2)

